%!TEX root = ../dokumentation.tex

\chapter{Einleitung}
\begin{displayquote}[Max Planck 1939\cite{Herneck:1976}]
    Wenn man genau ermittelt hat, wie viele Körner irgend ein Sandhaufen enthält, so ist das eine ungeheure Leistung; aber man kann mit dieser Zahl nichts weiter anfangen, weil sie nur eine ganz spezielle Bedeutung besitzt.
\end{displayquote}

Auch wenn diese Aussage von Max Planck im Bezug auf Bachelorarbeiten eher im Bauingenieurswesen, als in der Softwareentwicklung zutrifft, so stellt sie doch eine grundlegende Aussage zur Bewertung von wissenschaftlichen Arbeiten dar. Dieser Blickwinkel ist nicht nur für die Bachelorarbeit relevant, sondern zieht sich durch das gesamte Arbeitsleben. 

Die Kultusministerkonferenz beschreibt den Bachelorabschluss in ihrer Veröffentlichung zur Akkreditierung von Bachelor- und Masterstudiengängen aus dem Jahr 2010 als den ersten berufsqualifizierende Abschluss, der durch ein Studium erreicht werden kann. Die qualitätssichernde Komponente für das Erreichen des Bachelorabschlusses ist dabei die Bachelorarbeit. Sie soll sicherstellen, dass der Absolvent oder die Absolventin in der Lage sind, mit wissenschaftlichen Mitteln ein gegebenes Problem fristgerecht zu lösen.\cite[Vgl.][S.4]{kmk:2010}



\section{Gegenstand der Arbeit}
Doch wie ist die Qualität einer Bachelorarbeit zu bestimmen? Ab wann gilt eine Bachelorarbeit als qualitativ genug um die Vergabe eines Bachelortitels zur Rechtfertigen? Zur Beantwortung dieser Fragen ist die Definition von Qualität in einem ersten Schritt auf den gesamten Erstellungsprozess einer Bachelorarbeit zu übertragen. 
Max Planck erklärt in einer Aufnahme aus dem Jahre 1939 sein Verständnis von wissenschaftlichen Arbeiten.
In seinen Ausführungen weist er darauf hin, dass \enquote{[…] der Wert einer Erkenntnis im Wesentlichen davon abhängt, ob ihr eine allgemeine Bedeutung innewohnt.}\cite{Herneck:1976}
Auch hier stellt sich die Frage, wie lässt sich eine allgemeine Bedeutung objektiv Messen? 

In der folgenden Arbeit soll diese Thematik erarbeitet werden. Neben den zuvor genannten, grundsätzlichen Fragestellungen zur Qualitätsfeststellung einer Bachelorarbeit, ist die generelle Frage zu klären, ob sich die Qualität einer Bachelorarbeit rein objektiv Messen lässt. 

\section{Abgrenzung}
Diese Arbeit stellt einen Leitfaden für die qualitative Bewertung von Bachelorarbeiten dar und soll dadurch aufzeigen, wie die allgemeine Qualität dieser gesteigert werden kann. Sie ist jedoch nicht als Anleitung zum Schreiben einer Bachelorarbeit oder Handbuch für die Techniken zum Erstellen dieser zu verstehen. Für diesen Zweck existiert ausreichend Literatur, wie zum Beispiel das Buch \enquote{wissenschaftliches Arbeiten} von Axel Bänsch.\cite{Baensch:2013}

\chapter{Grundlagen und Stand der Forschung}
Wie in der Veröffentlichung der Kultusministerkonferenz dargestellt, soll die Bachelorarbeit zeigen, dass man in der Lage ist mit wissenschaftlichen Mitteln innerhalb eins gegebenen Zeitraums ein Ergebnis zu erarbeiten. Für die Bewertung einer solchen Arbeit stehen dadurch drei Hauptkriterien zur Verfügung. Die wissenschaftliche Arbeit, die Einhaltung des Termins und das Ergebnis. Da der Fokus dieser Arbeit auf der qualitativen Betrachtung der Bachelorarbeit liegt, ist zudem zu definieren, wie Qualität in diesem Zusammenhang zu verstehen ist.

\section{Was ist Qualität}
Der Begriff Qualität findet in jedem Themengebiet seinen Platz. In der \ac{ISO} Norm 9000 wird Qualität als \enquote{Grad in dem einem Objekt innewohnende Eigenschaften die Anforderungen erfüllen}\cite[Vgl.][Kapitel 3.6.2]{iso9000:2015} beschrieben. Das bedeutet, dass die Qualität an den Eigenschaften gemessen wird, die ein Objekt besitzt und nicht die, die ihm zugeordnet werden. Dabei können Objekte sowohl materiell als auch imaginär sein. Die Anforderungen werden definiert als \enquote{Bedürfnis oder Erwartung, das explizit genannt, impliziert oder obligatorisch ist}\cite[Vgl.][Kapitel 3.6.4]{iso9000:2015}. Somit ist Qualität nicht nur auf objektiv messbare Kriterien beschränkt, sondern kann auf Basis jeglicher, zuvor definierter Aspekte, bewertet werden. Im Bezug auf Bachelorarbeiten ist somit in einem ersten Schritt zu definieren, was die innewohnenden Eigenschaften sind. Daraufhin ist zu definieren, an welchen Kriterien diese Eigenschaften gemessen werden und wie diese Ergebnisse interpretiert werden müssen. Aus der Gesamtheit aller Aspekte und deren Gewichtung lässt sich die Qualität bewerten.

\section{Wissenschaftliches Arbeiten}
Das wissenschaftliche Arbeiten beschränkt sich nicht nur auf die reine Ausformulierung der Bachelorarbeit. Sie ist der gesamte Prozess der Erstellung. Genau an dieser Stelle greifen Qualitätsmanagement und wissenschaftliches Arbeiten ineinander. Betrachtet man die Definition eines Prozesses in der \ac{ISO} 9001 aus dem Jahr 2015, so ist dieser ein Satz von zusammenhängenden oder ineinandergreifenden Aktivitäten, die aus einem Input ein vorgegebenes Ergebnis erzeugen.\cite[Vgl.][]{iso9001:2015}
Der Input sind bei der wissenschaftlichen Arbeit zum einen die verwendete Literatur und zum anderen alle Ressourcen, die für die Forschung verwendet werden. Das Ergebnis ist die neue Erkenntnis, die durch die Arbeit erreicht wird. Materiell betrachtet auch das Dokument als solches. Dazwischen geschieht eine Abfolge von Aktivitäten, die im Folgenden beschrieben werden und deren messbare Eigenschaften hervorgehoben werden.

\subsection{Vorbereitende Maßnahmen}
Es beginnt bereits mit der Vorbereitung auf die Arbeit. Noch bevor das Thema definiert ist, sollte man mit seiner Arbeitsumgebung vertraut sein. Dazu gehört auf der einen Seite das Einarbeiten in die Funktionen der eigenen Schreibumgebung und Literaturverwaltung, auf der anderen Seite aber auch die Kenntnis über die Möglichkeiten zur Literaturrecherche im Umfeld der Hochschule.\cite[Vgl.][S. 43]{Baensch:2013} Dies sind keine messbaren Faktoren für eine gute Arbeit, sorgen jedoch dafür, dass zum Zeitpunkt der Ausarbeitung keine Zeit mehr in diese Grundlagen investiert werden muss.
Im Gegensatz dazu ist das Thema der Arbeit maßgeblich an der Qualität der Arbeit beteiligt. Es steckt durch seinen Inhalt und die Formulierung den Inhalt und somit auch die Erwartungen des Lesers ab. Bereits an ihm kann bewertet werden, ob die Inhalte des Studiums entsprechend abgedeckt werden. So deckt sich eine Bachelorarbeit im Bereich IT-Controlling sicherlich mit dem BWL-Modul im ersten Semester des Studiums der angewandten Informatik, bietet aber sonst keine Schnittpunkte mit den Inhalten des Studiums. Des Weiteren ist zu prüfen, ob die Thematik der Arbeit nicht bereits vollständig durch Literatur abgedeckt ist. Dieser Aspekt zeigt sich im Laufe der Arbeit durch eine fehlende oder sehr geringe Eigenleistung. Ein drittes, indirekt messbares Kriterium ist der Umfang des zu bearbeitenden Themas. Eine Bewertung ist bei Abschluss der Arbeit dadurch möglich, dass die inhaltliche Qualität mit dem Einhalten des Zeitrahmens verglichen wird. Weist eine Arbeit Defizite bei Inhalt auf oder wird der Zeitrahmen trotz ausreichend investierter Zeit überzogen, so kann es sein, dass das Thema zu allgemein formuliert war.\cite[Vgl.][S. 46 f.]{Baensch:2013} Es zeigen sich im Laufe dieser Arbeit jedoch noch weitere Gründe für ein solches Ergebnis.
Im Rahmen der Entwicklung des Titels der Arbeit entsteht gleichzeitig auch die Festlegung der zu behandelnden Themen. Diese ist im nächsten Schritt wichtig für die Auswahl der Literatur.

\subsection{Auswahl geeigneter Literatur}

Der nächste Schritt ist die Literaturrecherche. Auch hier kommt ein Teil der zu Beginn zitierten Aussage von Max Planck zum Tragen. Allein die Quantität ist nicht ausschlaggebend. Die Literatur sollte zudem auch qualitativ angemessen sein. Das bedeutet, dass diese ebenfalls ernst zu nehmend, kompetent und nachvollziehbar sein müssen. Dies kann auch für Artikel aus Tageszeitungen zutreffen, gleichzeitig aber auch Bücher, die nur den Anschein einer Wissenschaftlichkeit erwecken, ausschließen. Zudem ist, wenn möglich, fachspezifische Literatur zu verwenden.\cite[Vgl.][S. 7 f.]{Baensch:2013} Zur Bewertung kann somit ein Vergleich gezogen werden, zwischen den Themen, die in der Arbeit behandelt werden und den verwendeten Quellen. Interessant sind dabei die Fragestellungen: Wird jedes Themengebiet durch Quellen abgedeckt? Sind diese Quellen von wissenschaftlicher Relevanz? Decken die Quellen den aktuellen Forschungsstand ab, beziehungsweise sind sie der zeitlichen Einordnung des Themas angemessen?

\subsection{Schreiben der Arbeit}

Ist die Literaturrecherche abgeschlossen, beginnt die Umsetzung der Erstfassung. Sie sollte bereits alle inhaltlichen Elemente enthalten. Diese Aktivität ist maßgeblich für das subjektive Qualitätsempfinden des Lesers der Arbeit verantwortlich. Kriterien sind hierbei die Struktur der Arbeit, die einen klaren roten Faden erkennen lassen muss, sowie eine klare Argumentationsstruktur, um alle Aussagen in der Arbeit nachvollziehbar zu machen.\cite[Vgl.][S. 59]{Baensch:2013}
Nicht nur die Argumentationskette muss deutlich formuliert sein, wissenschaftliche Arbeit zeichnet sich viel mehr in ihrer Gesamtheit dadurch aus, dass sie Ablenkungen vom Thema durch einen unpassend gewählten Schreibstil vermeiden und die Gedanken und Schlussfolgerungen des Autors prägnant und unmissverständlich vermitteln. Auch die korrekte Verwendung der Sprache fließt in den Stil mit ein, sodass auch die Einhaltung von Rechtschreib-, Grammatik- und Zeichennutzungsregeln der jeweiligen Sprache in die Bewertung mit einfließen.\cite[Vgl.][S. 25-27]{Baensch:2013} 
Neben dem Sprachstil ist zudem auch die Zitierregeln relevant. Nur wenn das gesamte fremde Gedankengut sauber gekennzeichnet ist, kann von einer wissenschaftlichen Arbeit geredet werden. Auch hierbei gibt es Kriterien, die die Qualität der Arbeit beeinflussen. Bänsch und Alewell sprechen hierbei von einer Adäquaten Zitierweise. Abweichungen davon können entweder eine \enquote{Unterzitierung} sein, welcher eine zu geringe Anzahl adäquater Quellen vorangeht, oder eine \enquote{Überzitierung}, bei der unnötige häufig zitiert wird. Eine Extremform der Überzitierung ist dabei die Kompilation, bei der die Arbeit (fast) ausschließlich aus Zitaten besteht.\cite[Vgl.][S.11 f.]{Baensch:2013}
Die Erstfassung berücksichtigt absichtlich nicht alle Vorgaben zur Form. Ihr folgen die Überbearbeitungsphase und die Reinfassung. Dies ist vergleichbar mit der prototypenbasierten Entwicklung bei Softwareprojekten. Hierbei stellt die Erstfassung den Prototypen dar, der die Funktionalität des Ergebnisses darstellt, aber noch nicht vollständig ausgereift ist. Dieser Prototyp wird dann iterativ überarbeitet, bis alle Anforderungen erfüllt sind.\cite[Vgl.][S. 363]{Liggesmeyer:2009}

\section{Relevanz des zeitlichen Ablaufs}
\begin{table}[H]
    \centering
    \begin{tabular}{p{1.5cm} p{9.5cm} p{2.7cm}}
        \textbf{Phase} & \textbf{Zentraler Inhalt} & \parbox[c]{2.7cm}{\textbf{Dauer~in \\ Wochen~(ca.)}} \\ \hline
        (1) & Allgemeine Literatur-/Materialsammlung & 1,5 \\
        (2) & Sichten/Ordnen des Materials und Erstellen einer \textbf{Arbeits}gliederung & 1,0 \\
        (3) & Gezielte weitere Literatur- und Materialsammlung, Schreiben der \textbf{Erst}fassung mit parallelem Lesen von Literatur und Anpassung der Gliederung  & 5,0 \\
        (4) & Überarbeitung und Erarbeitung der abgabefähigen Fassung & 1,0 \\
        (5) & Korrekturen, \textbf{Rein}chrift und Überarbeitung von Abbildungen, Verzeichnissen etc. & 0,5 \\
         & \textbf{Insgesamt} & \textbf{9,0}
    \end{tabular}
    \caption[Zeitplan für eine neunwöchige Bachleorarbeit]{Zeitplan für eine neunwöchige Bachleorarbeit\protect\cite[entnommen aus][]{Baensch:2013}}\label{{tab:zeitplanung BA}}
\end{table}
Während die Einteilung der verfügbaren Zeit für die Erstellung der Arbeit das Ergebnis beeinflusst, kann die Aufteilung der Zeit auf die einzelnen Aktivitäten nicht direkt für die Bewertung einer Arbeit herangezogen werden. Sie kann jedoch im Nachhinein zur Qualitätssicherung herangezogen werden. Wird für eine Aktivität zu wenig Zeit eingeplant, so zeigt sich dies in einer unzureichenden Ausarbeitung. Bei der Literaturrecherche, aber auch bei der Erstfassung zeigt sich dies durch eine \enquote{Unterzitierung}. In der Gliederung und der Ausformulierung resultiert aus einer zu geringen Zeit eine unvollständige Struktur oder Argumentation. Fehlt bei den qualitätssichernden Aktivitäten, wie dem Überarbeiten und der Reinschrift die Zeit, steigt die Zahl der formalen und sprachlichen Fehler. 

In \autoref{{tab:zeitplanung BA}} ist eine exemplarische Aufteilung des neunwöchigen Bearbeitungszeitraums für eine Bachelorarbeit dargestellt. In einem gewissen Grad sind diese Zeiten an das Thema, die Umstände und die eigene Arbeitsweise anzupassen. Da sich Fehlplanung in der ersten und vierten Phase jedoch stark auf das Ergebnis auswirken, sollten Änderungen gut überlegt sein.\cite[Vgl.][S. 49 f.]{Baensch:2013}

\section{Aspekte zur Beurteilung der Bachelorarbeit}
Das Ergebnis der Bachelorarbeit ist die Ausarbeitung an sich. In dieser sind alle Erkenntnisse und Vorgehensweisen zu dokumentieren. An der \ac{DHBW} Stuttgart wird für die Informatik der in Anhang \nameref{apx:DHBW-Teil1} gezeigte Bewertungsbogen für Bachelorarbeiten eingesetzt. Dieser trennt die Bewertung in die inhaltliche Bearbeitung und die wissenschaftliche Arbeit. Die inhaltliche Bearbeitung fasst die betrieblich relevanten Aspekte zusammen. Dazu gehören die fachliche Bearbeitung, der Einsatz von Fachwissen und Methoden um Lösungsansätze zu erarbeiten. Zudem wird aufgrund des Praxisbezugs auch die Umsetzbarkeit und die Wirtschaftlichkeit des Ergebnisses berücksichtigt. 
Im wissenschaftlichen Teil der Bewertung kommen dann die strukturierte Vorgehensweise, lückenlose Dokumentation und die Verwendung von adäquater Literatur zum Tragen. 
Beide Teile werden etwa gleich gewichtet. Die Bewertung erfolgt auf Basis einer Skala von 0\% bis 100\% multipliziert mit einer Gewichtung, sodass in Summe maximal 100 Punkte erreicht werden können. Die Vergabe der Prozentpunkte geschieht dabei auf Basis der subjektiven Einschätzung des Prüfers und der Einordnung der Einschätzung anhand von vorgegebenen Definitionen. 
Durch dieses Schema können auch unerfahrenere Betreuer eine neutrale Beurteilung erstellen. Die einzelnen zu prüfenden Aspekte sind vorgegeben und für jeden dieser Aspekte ist beschrieben, welche Anforderungen für eine entsprechende Einordnung der Leistung gestellt werden. 
Zusätzlich zu diesem Schema müssen die Betreuer das in Anhang \nameref{apx:DHBW-Teil2} gezeigte Dokument ausfüllen und ihre Bewertungen begründen. 

Da diese Vorlage jedoch nur in der \ac{DHBW} Stuttgart für Bachelorarbeiten im Bereich Informatik genutzt werden, können die Kriterien für die Bewertung einer Arbeit in anderen Bereichen andere sein. So findet in der Fakultät Sozialwesen keine Bewertung auf Basis einer Skala statt, sondern durch eine binäre ja/nein Bewertung einzelner Kriterien. Diese sind dabei jedoch granulärer formuliert.\cite[Vgl.][]{DHBW-Sozial:2014}

Beide Bewertungsbögen greifen dabei sowohl die subjektiven, als auch die objektiven Bewertungskriterien auf. Dies spiegelt auch die Problematik beim Messen von Qualität wieder. Die Kriterien zur qualitativen Bewertung hängen von der Zielgruppe und dem Inhalt der Arbeit ab.

