%!TEX root = ../dokumentation.tex

\chapter{Einleitung}
Die Kultusministerkonferenz beschreibt den Bachelorabschluss in ihrer Veröffentlichung zur Akkredituerung von Bachelor- und Masterstudiengängen aus dem Jahr 2010 als den ersten berufsqualifizierende Abschluss, der durch ein Studium erreicht werden kann. Die qualitätssichernde Komponente für das erreichen des Bachelorabschlusses ist dabei die Bachelorarbeit. Sie soll sicherstellen, dass der Absolvent oder die Absolventin in der lage sind, mit wissenschaftlichen Mitteln ein gegebenes Problem fristgerecht zu lösen.\cite[Vgl.][S.4]{kmk:2010}

Mehr Text


\section{Gegenstand der Arbeit}
Doch wie ist die Qualität einer Bachelorarbeit zu bestimmen? Ab wann gilt eine Bachelorarbeit als qualitativ genug um die Vergabe eines Bachelortitels zur Rechtfertigen? Zur beantwortung dieser Fragen ist die definition von Qualität in einem ersten Schritt auf den gesamten Erstellungsprozess einer Bachelorarbeit zu übertragen. 
In einer Aufnahme von Max Plank aus dem Jahre 1939, erklärt er sein Verständnis von wissenschaftlichen Arbeiten. In seinen Ausführungen weißt er darauf hin, dass "[..]  der Wert einer Erkäntnis im wesentlichen davon abhängt, ob ihr eine allgemeine Bedeutung innewohnt."\cite{Herneck:1976}
Auch hier stellt sich die Frage, wie lässt sich eine allgemeine Bedeutung objektiv Messen? 

In der folgenden Arbeit soll diese Thematik erarbeitet werden. Neben den zuvor genannten, grunsätzlichen Fragestellungen zur Qualitätsfeststellung einer Bachelorarbeit, ist die generelle Frage zu klären, ob sich die Qualität einer Bachelorarbeit reihn objektiv Messen lässt. 


\section{Grundlagen und Stand der Forschung}
Um die genannten Fragen zu beantworten, ist es notwendig die zu Grunde liegenden Anforderungen an eine Bachelorarbeit zu verstehen. Ebenso ist es notwendig den Begriff Qualität im Bezug auf dieses Thema zu definieren und die Messbarkeit dieser darzustellen.  
\subsection{Was ist Qualität}

\subsection{Ansprüche an eine wisschenschaftliche Arbeit}

\subsection{Stand der Forschung}


\subsection{Abgrenzung}
Diese Arbeit stellt einen Leitfaden für die qualitative Bewertung von Bachelorarbeiten dar und soll dadurch aufzeigen, wie die allgemeine Qualität dieser gesteigert werden kann. Sie ist jedoch nicht als Anleitung zum schreiben einer Bachelorarbeit oder Handbuch für die techniken zum Erstellen dierser zu verstehen. Für diesen Zweck existier ausreichend Literatur, wie zum Beispiel das Buch wissenschaftliches Abreiten von Axel Bänsch\cite{Baensch:2013}.

- Grundlagen erarbeiten

- Welche Vorkentnisse werden benötigt?

- Erläutern der relevanten Theorien und Hintergründe

- Stand des verfügbaren Wissens (methoden, Hypothesen)

- Abgrenzung


\section{Schluss}

- Zusammenfassung

- Fazit

- Ausblick
