%!TEX root = ../dokumentation.tex

\chapter{Hauptteil}
Dieses Kapitel wird im Rahmen dieser Hausarbeit nicht gefüllt. Der Fokus liegt auf der Struktur und den Grundlagen.


\chapter{Schluss}
Im Rahmen dieser Arbeit wurde der Aufbau und die einzelnen Kriterien zur qualitativen Beurteilung einer Bachelorarbeit beleuchtet. Sie folgt dabei den Regeln für wissenschaftliches Arbeiten und ist der erste berufsqualifizierende Abschluss, der durch ein Studium erreicht werden kann.
Der wissenschaftliche Anspruch dieser Arbeit spiegelt sich auch in der Bewertung dieser wieder. Der gesamte Entstehungsprozess der Arbeit folgt einem vorgegebenen Schema, das mit dem Formulieren eines Themas und einer zu untersuchenden Fragestellung beginnt, das Benutzen von angemessenen Literatur- und Materialquellen und deren korrekte Zitierung beinhaltet und in einem selbst erarbeiteten und kritisch hinterfragtem Ergebnis mündet. Diese Elemente bieten sowohl objektive als auch subjektive Kriterien zur Feststellung ihrer Qualität. Im Vergleich zwischen zwei Fakultäten der \ac{DHBW} Stuttgart hat sich ergeben, dass die Schwerpunkte und Art der Bewertung je nach Themengebiet unterschiedlich ist, dabei aber trotzdem alle Aspekte abdeckt. 

Um die Qualität einer Bachelorarbeit zu bestimmen, werden auch in den verschiedenen Bereichen die gleichen Aspekte betrachtet. Im Endeffekt wird jedoch diese Bewertung auf einen Durchschnitt reduziert und als Note im Zeugnis vermerkt. Dadurch geht bei einer späteren Betrachtung jedoch die Vergleichbarkeit verloren. Am Beispiel des Bewertungskatalogs für die Informatik ist gut zu sehen, dass eine praktisch sehr gut bewerte Arbeit trotz eklatanter Mängel in der wissenschaftlichen Ausarbeitung eine gute Note haben kann. Genauso kann eine schlechte Umsetzung durch eine sehr gute Literaturrecherche und Einhaltung aller Formalien ebenfalls ausgeglichen werden. Es stellt sich die Frage, ob die finale Auszeichnung einer Arbeit durch eine einzelne Note die Qualität einer Arbeit widerspiegeln kann. 
 
